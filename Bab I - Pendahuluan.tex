% ==========================================
% BAB I PENDAHULUAN
% ==========================================
\chapter{PENDAHULUAN}
\label{chap:pendahuluan}
% --- Latar Belakang ---
\section{Latar Belakang}
Ancaman siber saat ini telah berevolusi dari malware tradisional berbasis file menjadi serangan yang lebih canggih dan tersembunyi, seperti fileless \textit{malware} dan \textit{spyware}. Malware jenis ini dirancang khusus untuk menghindari deteksi dengan memanfaatkan fitur-fitur sah dari sistem operasi, seperti PowerShell dan Windows Management Instrumentation (WMI), untuk menjalankan kode berbahaya langsung di memori tanpa menulis file apapun ke disk. Karena tidak memiliki file yang dapat dideteksi oleh \textit{antivirus} berbasis tanda tangan (signature-based), serangan ini seringkali luput dari pantauan sistem keamanan tradisional.

Kecanggihan ini membuat fileless malware menjadi ancaman yang sangat berbahaya bagi perusahaan manapun karena kemampuannya untuk bertahan lama dan menghindari solusi \textit{antivirus} yang ada. Menurut penelitian, \textit{anti-spyware} modern masih memiliki kelemahan signifikan, bahkan terhadap malware "lama" yang dimodifikasi dengan trik baru. Misalnya, sebuah studi pada tahun 2023 menemukan bahwa hampir separuh dari 12 mesin \textit{antivirus} yang diuji hanya mampu mendeteksi kurang dari setengah varian malware yang disamarkan. Penelitian lain juga menunjukkan bahwa metode seperti enkripsi, injeksi proses, dan penambahan data sampah ke file eksekusi (\textit{junk data}) terbukti sangat efektif dalam menghindari deteksi. Teknik ini bahkan dapat membingungkan \textit{antivirus} gratis dan berbayar, membuat mereka gagal mendeteksi atau mengkarantina file yang berbahaya.

Situasi ini semakin diperparah dengan temuan bahwa banyak alat keamanan, termasuk \textit{anti-spyware}, sering kali gagal menganalisis kode yang dikemas menggunakan alat populer seperti PyInstaller untuk bahasa \textit{Python}. Hal ini terjadi karena \textit{antivirus} tidak dapat memahami konten kode bita \textit{Python (Python bytecode)}, yang secara efektif menyamarkan skrip berbahaya dari analisis statis. Kurangnya deteksi yang efektif oleh solusi keamanan yang ada, termasuk \textit{anti-spyware} dan EDRs (\textit{Endpoint Detection and Response}), menciptakan celah besar yang dieksploitasi oleh kelompok penjahat siber dan APTs (\textit{Advanced Persistent Threats}), yang semakin sering menggunakan skrip \textit{fileless PowerShell} untuk menghindari pertahanan.


% --- Rumusan Masalah ---
\section{Rumusan Masalah}
Berdasarkan latar belakang di atas, maka rumusan masalah dalam penelitian ini adalah:
\begin{enumerate}
\item	Bagaimana kapabilitas sistem \textit{anti-spyware} dalam mendeteksi aktivitas tersembunyi dari perilaku \textit{spyware mode stealth} pada lingkungan uji terkontrol?
\item	Bagaimana pendekatan paling efektif untuk menyisipkan \textit{spyware mode stealth} ke sistem?
\item	Bagaimana teknik penyamaran (obfuscation) dan metode eksekusi berbasis memori (in-memory execution) mempengaruhi kemampuan deteksi produk \textit{anti-spyware} saat ini?
\item Apa saja celah keamanan dan kelemahan spesifik yang ditemukan pada produk \textit{anti-spyware} dan EDR ketika dihadapkan pada serangan spyware kustom yang dirancang untuk menghindari deteksi?
\end{enumerate}

% --- Tujuan ---
\section{Tujuan}
Secara umum, tujuan dari pelaksanaan tugas akhir ini adalah untuk mengukur dan mengevaluasi efektivitas produk \textit{anti-spyware} dan EDR komersial dalam mendeteksi \textit{spyware mode stealth} yang dikembangkan dengan teknik-teknik penghindaran deteksi modern.

Secara spesifik, tujuan yang ingin dicapai adalah:
\begin{enumerate}
    \item Menganalisis dan mengidentifikasi teknik-teknik evasi yang paling efektif digunakan oleh \textit{spyware} untuk menghindari deteksi dari \textit{anti-spyware} dan EDR.
    \item Mengembangkan sebuah prototipe \textit{spyware mode stealth}, termasuk \textit{reverse shell fileless PowerShell}, yang menggabungkan teknik-teknik evasif yang telah diidentifikasi.
    \item Melakukan pengujian komparatif prototipe \textit{spyware} pada sejumlah produk \textit{anti-spyware} dan EDR komersial untuk mengevaluasi tingkat keberhasilan dan kegagalan deteksi.
    \item Mendokumentasikan dan mempublikasikan celah keamanan yang ditemukan pada produk \textit{anti-spyware}, serta menyusun rekomendasi mitigasi untuk pengembangan sistem pertahanan yang lebih adaptif dan tangguh.
\end{enumerate}

kriteria keberhasilan dari pelaksanaan tugas akhir ini adalah:

\begin{itemize}
    \item Prototipe \textit{spyware} berhasil dikembangkan dengan setidaknya dua teknik evasif (misalnya, enkripsi dan obfuscation) dan mampu menghindari deteksi oleh salah satu produk \textit{anti-spyware} yang diuji.
    \item Hasil pengujian menunjukkan bahwa ada perbedaan signifikan dalam tingkat deteksi antara format skrip dan \textit{anti-spyware} yang berbeda.
\end{itemize}

Kedua kriteria tersebut menjadi indikator utama untuk menilai efektivitas penelitian, serta menunjukkan sejauh mana solusi yang ditawarkan mampu mengungkap kelemahan sistem keamanan siber saat ini. Pencapaian terhadap kriteria ini diharapkan dapat menjadi dasar pertimbangan dalam peningkatan kapabilitas deteksi \textit{anti-spyware} dan EDR di masa depan

% --- Batasan Masalah ---
\section{Batasan Masalah}
Batasan masalah dalam pelaksanaan tugas akhir adalah sebagai berikut: 
\begin{enumerate}
    \item Penelitian ini hanya berfokus pada pengujian kapabilitas deteksi \textit{anti-spyware} terhadap aktivitas awal (\textit{Initial Access}) yang dilakukan oleh \textit{spyware mode stealth} pada sistem operasi \textit{desktop}.	
    \item Evaluasi hanya mencakup perangkat anti \textit{spyware} yang dipilih sesuai kriteria penelitian, tidak membandingkan seluruh produk anti malware yang ada.
    \item Aktivitas \textit{malware} yang diujikan dibatasi secara ketat pada tahap penyusupan, enkripsi \textit{payload}, \textit{fileless execution}, dan upaya \textit{persistence}.
    \item Tugas akhir ini dikerjakan secara kelompok dengan anggota penelitian sebagai berikut: 
    \begin{itemize}
        \item Nathaniel Liady
        \item M. Kasyfil Aziz
        \item Audra Zelvania Putri Harjanto
        \item Khayla Belva Annandira
\end{enumerate}

% --- Metodologi Pengerjaan TA ---
\section{Metodologi}
Metodologi penelitian yang digunakan adalah \textit{Design Research Methodology} (DRM) dikenalkan oleh \cite{Blessing2009DesignResearchMethodology}. DRM dibuat dengan tujuan supaya riset dilakukan dengan lebih efektif dan efisien. Metodologi ini terdiri dari empat tahap utama yaitu Research Clarification (RC), Descriptive Study I (DS-I), Perspective Study (PS), dan Descriptive Study II (DS-II). Berikut ini adalah gambaran dari kerangka kerja DRM. 
    \begin{figure}[h]
    \centering
    \includegraphics[width=0.8\textwidth]{image/Design-Research-Methodology-Framework-Blessing-and-Chakrabarti-2009.png}
    \label{fig: \textit{Design Research Methodology}}
    
    Gambar 1.1 \textit{Design Research Methodology} Framework 
    \end{figure}
\begin{enumerate}
\item \textit{Research Clarification (RC)}
    
    Fase ini akan dimulai dengan identifikasi masalah utama: adanya celah yang signifikan antara teknik serangan siber modern, khususnya \textit{spyware mode stealth}, dan kemampuan deteksi solusi keamanan yang ada. Pengumpulan data awal akan dilakukan melalui tinjauan literatur komprehensif, termasuk laporan industri, artikel akademis, dan berita, untuk memahami lanskap ancaman dan teknik evasif yang digunakan untuk menghindari deteksi \textit{anti-spyware} dan EDR. Hasil dari fase ini adalah rumusan masalah yang jelas dan terperinci, yang akan menjadi landasan untuk seluruh penelitian.
\item \textit{Descriptive Study I (DS-I)}
    
    Pada fase ini, analisis mendalam terhadap masalah yang telah dirumuskan akan dilakukan dengan mengumpulkan data dan informasi yang relevan. Ini mencakup analisis rinci mengenai teknik-teknik evasi canggih, seperti penggunaan \textit{PowerShell} dan obfuscation, untuk memahami bagaimana ancaman ini bekerja dan mengapa mereka sulit dideteksi. Selain itu, studi-studi terdahulu yang telah menguji bypass \textit{antivirus} akan dikaji untuk mendapatkan wawasan mengenai metode dan potensi hasil. Sebagai bagian penting dari fase ini, alur kerja atau arsitektur teknis dari serangan \textit{spyware} akan disusun, yang akan menjelaskan fase-fase seperti \textit{Initial Access, Establish Foothold, Persistence, Data Collection}, dan \textit{Exfiltration}. Diagram alur yang telah ada akan menjadi representasi visual dari arsitektur ini.
    \item \textit{Perspective Study (PS)}

    Fase ini merupakan inti dari DRM, di mana solusi untuk masalah yang telah didefinisikan akan dirancang dan dikembangkan. Artefak yang akan dibuat adalah prototipe \textit{spyware mode stealth} dengan fungsi-fungsi spesifik seperti \textit{keylogging} dan \textit{screen capture}. Desain prototipe akan mengintegrasikan teknik evasi yang telah diidentifikasi pada fase sebelumnya, seperti enkripsi \textit{payload} dan penggunaan PowerShell untuk menjalankan kode langsung di memori. Berbagai modul, termasuk Packer untuk menyamarkan \textit{payload}.
    \item \textit{Descriptive Study II (DS-II)}
    
    Fase terakhir ini bertujuan untuk menguji dan mengevaluasi efektivitas solusi yang telah dikembangkan. Serangkaian pengujian eksperimental akan dilakukan dalam lingkungan virtual yang terisolasi untuk mengukur tingkat deteksi berbagai produk \textit{anti-spyware} terhadap prototipe \textit{spyware}. Hasil pengujian akan dianalisis untuk mengidentifikasi teknik evasi mana yang paling efektif dan mengapa produk keamanan tertentu gagal atau berhasil dalam mendeteksi ancaman. Analisis ini akan memvalidasi temuan awal dan memberikan kontribusi nyata pada pemahaman tentang kerentanan sistem keamanan modern, yang akan menjadi dasar untuk rekomendasi perbaikan dan penelitian lebih lanjut.
\end{enumerate}