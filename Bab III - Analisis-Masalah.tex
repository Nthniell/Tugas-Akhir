% ============================================================================================
% BAB III ANALISIS MASALAH
% Pembagian subbab tidak rigid dan dapat bervariasi. Bab ini minimal berisi analisis kebutuhan
% fungsional dan nonfungsional, analisis berbagai alternatif solusi yang dapat ditawarkan, dan
% metode pemilihan solusi yang diusulkan.
% ============================================================================================
\chapter{ANALISIS MASALAH}
\label{chap:analisis-masalah}

\section{Analisis Kondisi Saat Ini}
Kondisi keamanan siber saat ini ditandai dengan evolusi ancaman yang signifikan, bergerak dari \textit{malware} tradisional berbasis \textit{file} menuju serangan yang lebih canggih dan tersembunyi, seperti \textit{fileless malware} dan \textit{spyware mode stealth}. Evolusi ini menciptakan celah kritis dalam kemampuan deteksi sistem \textit{anti-malware (AV)} dan \textit{Endpoint Detection and Response (EDR)} yang masih berpegangan pada mekanisme pertahanan konvensional.

Secara konseptual, serangan \textit{spyware} modern yang berfokus pada \textit{Initial Access} dan \textit{Packer} melibatkan beberapa komponen utama: Target (sistem yang diserang), \textit{Packer/Dropper} (Artefak \textit{spyware} yang disamarkan), \textit{Anti-Malware/EDR} (Sistem pertahanan), dan \textit{Server} (Pengumpul informasi, seperti yang digambarkan dalam alur \textit{General Flow}).

\subsection{Masalah Kerentanan dan Ketertinggalan Anti-Malware Konvensional}
Kondisi keamanan siber saat ini ditandai dengan evolusi ancaman yang signifikan, bergerak dari \textit{malware} tradisional berbasis \textit{file} menuju serangan yang lebih canggih dan tersembunyi, seperti \textit{fileless malware} dan \textit{spyware mode stealth}. Evolusi ini menciptakan celah kritis dalam kemampuan deteksi sistem \textit{anti-malware (AV)} dan \textit{Endpoint Detection and Response (EDR)} yang masih berpegangan pada mekanisme pertahanan konvensional.


\subsection{Identifikasi Masalah Pengguna}
\lipsum[5]
\subsection{Kebutuhan Fungsional}
\lipsum[6]
\subsection{Kebutuhan Nonfungsional}
\lipsum[7]

\section{Analisis Pemilihan Solusi}
\subsection{Alternatif Solusi}
\lipsum[8]
\subsection{Analisis Penentuan Solusi}
\lipsum[9]